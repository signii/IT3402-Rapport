\section{\textcolor[HTML]{D32F2F}{Forarbeid}}

%domenekunnskap
Når man jobber med brukersentrert design er det nettopp brukeren som skal være i sentrum. Beslutninger skal tas med brukeren i tankene, og ingen ting skal overlates til tilfeldighetene. Gruppen fulgte derfor seks prinsipper for brukersentrert design:
\begin{enumerate}
    \item Designet skal være basert på̊ en eksplisitt forstå̊else av brukerne, deres oppgaver og omgivelser
    \item Brukere skal være involvert i hele design- og utviklingsprosessen
    \item Designet er drevet av brukersentrert evaluering
    \item Prosessen er iterativ
    \item Designet adresserer hele brukeropplevelsen
    \item Designteamet består av multidisiplinære perspektiv
\end{enumerate}

\noindent Ved å ha disse seks punktene i fokus sørget gruppen for at brukeren ble holdt i fokus gjennom hele prosessen. I tillegg til de seks punktene kan man også klassifisere brukersentrerte aktiviteter i fire kategorier. Disse kategoriene er: forståelse og spesifikasjon av brukskontekst; spesifikasjon av brukernes krav; produksjon av designløsninger som møter brukerkravene; og evaluering av design. Disse fire kommer i tillegg til planlegging og implementering.

\subsection{Teknologiundersøkelse}

\subsection{Valg av produkt}
For å skaffe et overblikk over hvilke behov kunder på Sirkus Shopping har, tok gruppen turen til kjøpesenteret for å gjøre feltobservasjoner. Her ble det observert kjønn, alder, grupperinger og andre karakteristikker på kundene. Det ble også gjort observasjoner på hendelser som utspant seg i butikkene og i lokalene utenfor butikkene. 
\\\\
Observasjon er en nyttig måte å samle inn data på. Brukere kan enten observeres direkte av en observatør i et laboratorium, eller observatøren kan observere brukere i felt \cite[s.~253]{preece}. Gruppen valgte å observere brukere i felt i første runde. Grunnen til at feltobservasjoner er nyttig er at brukere sjelden får til å beskrive nøyaktig hva de gjør. En observatør kan legge merke til andre ting enn det brukeren tenker på som viktig, og så kan disse observasjonene i ettertid bekreftes eller avkreftes med et spørreskjema. 
\\\\
Observasjonene som ble gjort på kjøpesenteret hadde som mål å finne ut hvem som bruker senteret, hva de handler, og mønsteret de bevegde seg i. Gruppen så på kjønn, alder, gruppering (handlet sammen eller alene), handlemønster og hva som ble sagt. Alt ble notert ned sammen med tid på døgnet, siden gruppen tenkte det kunne være forskjell på når ulike grupper besøkte senteret.
Basert på observasjonene som ble gjort på Sirkus Shopping lagde gruppen to personas som beskrev de typiske kundene på senteret. Gruppen studerte observasjonene nøye, og kom opp med ulike ideer til produkter som personasene kunne ha nytte av. Ideene ble diskutert innad i gruppen, samt med studentassistenten gruppen fikk tilordnet. Etter flere idémyldringer ble gruppen enige om et produkt alle hadde god tro på. 

\subsection{Analyse av behov}
Brukeropplevelsen er viktig når man jobber med interaksjonsdesign. Brukeropplevelse beskrives som hvordan et produkt oppfører seg og blir brukt av personer i den virkelige verden (Preece et al, 2015, s. 12). Ethvert produkt som brukes av noen har en brukeropplevelse, eller UX som det forkortes på engelsk. Brukeropplevelse kan ikke designes, men man kan legge til rette for en god brukeropplevelse. Ved å analysere brukernes behov kan man få et bedre resultat enn hvis man setter i gang arbeidet med å utvikle noe uten å ha snakket med sluttbrukerne.
\\\\
%skrive noe om behovene
Brukbarheten til et produkt kan måles, ved å se på seks punkter som peker på ulike mål. Disse listes gjerne opp som spørsmål, og målet er å gi interaksjonsdesigneren bekreftelse på om produktet er brukervennlig eller ikke \cite[s.~19]{preece}. De seks punktene man ser på er hvor effektivt noe er å bruke(\textit{effectiveness}), hvor raskt det er å bruke (\textit{efficiency}), hvor trygt det er å bruke (safety), hvor nyttig det er (\textit{utility}), hvor enkelt det er å lære (\textit{learnability}), og hvor enkelt det er å huske (\textit{memorability}). Alle disse punktene er det lurt å ha i tankene når produktet utvikles, for å så ta de fram igjen når produktet skal testes. 

\subsection{Brukergruppe}
Da gruppen kom til analysefasen ble brukernes behov kartlagt. Gruppen hadde brukt tid på å finne ut hvem brukerne var, og hadde laget to personas som representerte to brukere. Disse var basert på observasjoner, analyse av målgruppe og segmenter og intervjuer. 

\subsection{Ideen bak produktet}
Gjennom observasjoner og intervjuer så gruppen det at det var et ønske om en enklere handleopplevelse. Mange kunder opplever det som slitsomt å måtte bære rundt på varene de hadde kjøpt, og noen ønsket også å kunne se alle typer av et produkt før de bestemte seg for hvilken de ville kjøpe. Eksempelvis var det noen kunder som ønsket å kunne se alle genserne på senteret før de kjøpte en av dem. De ønsket ikke å gå innom flere butikker som hadde noe av det samme utvalget, men heller se og prøve alle genserne på samme sted.
\\\\
Et annet punkt som ble nevnt av intervjuobjektene var oversiktlighet på senteret. Mange sentre har en infotavle, men det er få som tar seg tid til å studere denne nøye. Få sentre er også lagt opp slik at butikker i samme kategori ligger ved siden av hverandre. Intervjuobjektene fortalte at det kunne være vanskelig å finne fram til butikkene de ville besøke dersom de hadde lite tid, og ønsket en løsning på dette. 
\\\\
Gruppen kom opp med en idé om en app hvor man har en virtuell handlekurv med alle produktene man ønsker å kjøpe. Når kunden finner et produkt hun eller han ønsker å kjøpe, scanner kunden en RFID i prislappen som gjør at produktet dukker opp i appen og kan legges i handlekurven. I appen velger kunden størrelse, og kan endre antall. I den virtuelle handlekurven vil kunden se en oversikt over produkter hun eller han har valgt seg ut, samt en totalsum på varene. 
\\\\	
Når kunden på slutten av handleturen ønsker å betale og ta med seg varene hjem vil ordren sendes til utleveringsstedet og kunden betaler for varene. Dette kan enten gjøres i appen, eller ved utleveringsstedet for kunder som ikke har aktivert betalingstjenester på smarttelefonen sin. Dersom kunden ikke vil handle varene med en gang, kan de legges til i en ønskeliste og lagres til senere. For eksempel kan en student gå rundt på senteret og legge produkter til i sin ønskeliste, og før jul dele denne med bestemor eller onkel slik at de kan gå inn på senteret og handle riktig produkt og størrelse uten problem.
\\\\
For å løse punktet om navigasjon inne på senteret kom gruppen opp med en idé om å ha streker på gulvet som indikerer en rute man kan følge. Det kan for eksempel være å ha en grønn rute for herrer. Da er det en linje i gulvet som går innom alle butikker som herrer har interesse for på senteret, som for eksempel klesbutikk for menn. Tilsvarende kan man ha en rute for sport, som går innom sportsbutikker på senteret. Disse rutene går det også an å ha tilpasset sesong. Rundt skolestart kan det for eksempel legges opp en rute som går innom alle butikkene hvor man kan handle ting til skolestart. For eksempel bokhandel, sportsbutikk og klesbutikk.

%bilde med masse postitlapper (ideer)


\section{\textcolor[HTML]{D32F2F}{Prosjektstyring}}

\subsection{Produkteier}

\subsection{Roller}
\subsubsection{Forsøksleder}
En testleder bør være nysgjerrig, tålmodig og ærlig, og ha ydmykhet ovenfor andres meninger og opplevelser\cite{brukskvalitetstesting}.

\subsection{Arbeidsstruktur}
Gruppen bestod av fire studenter, alle tilhørende informatikk. Siden alle studentene hadde ulik timeplan ble det opprettet en samtale på Facebook hvor gruppen kunne holde kontakt med hverandre. I utgangspunktet avtalte gruppen å møtes en gang i uken, men dette ble noen ganger endret avhengig av arbeidsoppgaver og frister. I tillegg til dette møtte studentene studentassistenten i emnet omtrent annenhver uke. 
\\\\
For å samle alle dokumenter i prosjektet på ett sted ble det valgt å bruke Google Drive. Her kan flere brukere jobbe samtidig på samme dokumenter, samtidig som man også kan laste opp eksterne filer. Dette ble blant annet brukt til å skrive rapport, samle informasjon, lage spørreundersøkelse og lagringsplass for plakat. 
\\\\
Prosjektets mål var å lære studentene om brukersentrert design i praksis. For at prosjektarbeidet skulle være strukturert som en iterativ designprosess var det nødvendig at oppgavene ble gjort i faser som ble repetert gjentatte ganger gjennom semesteret. Fasene kunne deles inn i fire deler; analyse, design, prototyping og testing. Når alle fasene var gjennomført begynte gruppen å gjøre fasene på nytt. 

\subsubsection{Utviklingsprosess}
For å finne et design som passet kundenes behov, ble det valgt å følge en iterativ prosess. Dette er nyttig når man skal utvikle et produkt for en brukergruppe, fordi man underveis kan endre og tilpasse til kundenes behov. En iterativ prosess består gjerne av fire faser; analyse, design, prototype og testing\cite{brukersentrert}. I første runde kan prototypene være svært primitive, low-fidelity, som for eksempel en papirprototype. Senere i prosessen kan man med fordel gjøre prototypen med avansert, for eksempel ved å lage en high-fidelity prototype\cite{paperprototype}.
\\\\
I tabell \ref{tab:prosess} er det vist en oversikt over aktivitetene gruppen gjorde i de ulike fasene av prosjektet.

%fyll inn mer her
\begin{table}[H]
    \caption{Ovrsikt over aktiviteter}
    \label{tab:prosess}
    \centering
    \begin{tabular}{|L{3em}| L{9em}|L{9em}|L{9em}|L{9em}|}
    \hline
        \rowcolor[HTML]{D32F2F}
        \textbf{\textcolor{white}{Fase}} & \textbf{\textcolor{white}{Forstå}} & \textbf{\textcolor{white}{Spesifisere}} &  \textbf{\textcolor{white}{Designe}}& \textbf{\textcolor{white}{Evaluere}}\\
        \rowcolor[HTML]{E6E6E6}
        0 & Observasjoner i felt, personas og scenarier & Konsept og ideer & - & Innsamlede data\\
        1 & Intervjuer av brukere, spørreundersøkelse & Idé og beskrivelse av app & Papirprototype & Brukertester av papirprototype\\
        \rowcolor[HTML]{E6E6E6}
        2 & Intervju med butikkansatte & Justere app, behov & Redesign av papirprototype & -\\
        3 & - & Behov & Wireframe-prototype & Eyetracking, brukertester av wireframe-prototype\\
        \rowcolor[HTML]{E6E6E6}
        4 & - & - & Videre design & Prosess\\
        \hline
    \end{tabular}
\end{table}
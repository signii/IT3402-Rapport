\section{\textcolor[HTML]{D32F2F}{Prosjektstyring}}

\subsection{Produkteier}

\subsection{Roller}

\subsection{Arbeidsstruktur}
Gruppen bestod av fire studenter, alle tilhørende informatikk. Siden alle studentene hadde ulik timeplan ble det opprettet en samtale på Facebook hvor gruppen kunne holde kontakt med hverandre. I utgangspunktet avtalte gruppen å møtes en gang i uken, men dette ble noen ganger endret avhengig av arbeidsoppgaver og frister. I tillegg til dette møtte studentene studentassistenten i emnet omtrent annenhver uke. 
\\\\
For å samle alle dokumenter i prosjektet på ett sted ble det valgt å bruke Google Drive. Her kan flere brukere jobbe samtidig på samme dokumenter, samtidig som man også kan laste opp eksterne filer. Dette ble blant annet brukt til å skrive rapport, samle informasjon, lage spørreundersøkelse og lagringsplass for plakat. 
\\\\
Prosjektets mål var å lære studentene om brukersentrert design i praksis. For at prosjektarbeidet skulle være strukturert som en iterativ designprosess var det nødvendig at oppgavene ble gjort i faser som ble repetert gjentatte ganger gjennom semesteret. Fasene kunne deles inn i fire deler; analyse, design, prototyping og testing. Når alle fasene var gjennomført begynte gruppen å gjøre fasene på nytt. 
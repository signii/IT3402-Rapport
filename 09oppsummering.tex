\section{\textcolor[HTML]{D32F2F}{Oppsummering}}
\label{oppsummering}
Det viktigste vi lærte i prosjektet var hvor nyttig det var med brukertester.
Brukere tenker annerledes enn de som lager produktene, og det er viktig å ikke lage noe med egne behov i sentrum. Ved å hyppig brukerteste produktet fikk vi mange gode tilbakemeldinger underveis, og kunne rette oss etter det brukerne sa. 
%vite hvilke behov som finnes
%tenke høyt under test
Gruppen har samarbeidet godt gjennom prosessen, og hatt en god arbeidsflyt. Det var ganske enkelt å bli enige om konsept, noe som gjorde det lett å motivere seg for arbeidet når alle hadde tro på konseptet. Arbeidsoppgavene har vært godt fordelt på alle gruppemedlemmene, og vi var tidlig ute med å begynne på rapporten. Det har gjort arbeidet i slutten av prosjektet enklere enn om vi ikke hadde dokumentert noe tidlig.

\subsection{Refleksjon om tverrfaglige grupper}
I gruppen som ble satt sammen var alle medlemmene informatikkstudenter. Gruppen ble dermed ikke like tverrfaglig som mange av de andre gruppene i emnet, men det fungerte stort sett greit. Det kunne selvfølgelig vært en fordel å for eksempel ha en student fra produktdesign når det skulle lages og printes poster, men gruppen løste oppgaven godt selv uten ressursene som mange andre grupper hadde.

\subsection{Retrospekt}
Dersom vi skulle gjort prosjektet en gang til er det noen ting vi ville gjort annerledes. For det første ville vi ha kjørt flere runder med observasjoner på senteret. Det kunne også vært spennende å holde fokusgrupper og få innspill fra brukere på den måten. Da hadde vi nok hatt behov for mer ressurser (tid, rom, kontakt med brukere), men det ville absolutt vært gjennomførbart. En annen ting vi kunne gjort annerledes er å ikke spisse inn så tidlig som vi gjorde. Vi kunne nok hatt nytte av en lenger idémyldringsprosess, men samtidig har prosjektet en ganske tett tidsplan og det kan ikke brukes for mye tid på hver fase.

\subsection{Videre arbeid med produkt}
Prototypen vi har kommet fram til skiller seg klart fra andre apper som finnes på markedet i dag, og vil gi brukerne en helt ny måte å handle på. Vi mener at dette absolutt kan være verdt å satse på. Det forutsetter at man får med seg handlesentrene på ideen, men vi tror at dette kan hjelpe mange kunder med å handle mer effektivt. På denne måten kan man sammenligne priser og produkter, og det blir også mulig for kunden å se antall på lager. Dette er noe mange brukere nevnte i spørreundersøkelsen vi gjorde, og det tyder på at dette er noe forbrukerne savner. Løsningen vi har laget vil også gjøre det raskt og lettvint for forbrukere på storhandel å handle. De trenger ikke bære på mange handleposer, eller gå rundt med stor handlevogn. Alle varene hentes til slutt, ferdig pakket, og kan bæres rett inn i bilen.
\\\\
Om butikkene vil se på dette som en god løsning er ikke sikkert, men vi tror at dersom man får kartlagt alle behov og finner en god løsning av alle aspekt vil dette kunne trekke mange kunder. Løsningen vi har kommet opp med krever en annen disponering av ressurser enn man ser i handlesentrene i dag. Med vår løsning vil man ikke trenge personell i kassen, men heller ute i butikken som kan hjelpe kundene. Det vil da være med å finne riktig størrelse, besvare spørsmål om produktet og gi råd til hva man burde velge. Dette er selvfølgelig uheldig for de ansatte, men for kundene vil dette kanskje gjøre at prisene blir lavere. Dersom butikkene ikke behøver å ha så mange ansatte på jobb, blir utgiftene lavere og da kan også produktene selges lavere. Nettopp det er ideen bak nettbutikker, som er en sterk konkurrent til handlesentrene. I dag går mange kunder først i en butikk og prøver et produkt, for å så handle produktet på nettet når de kommer hjem fordi de der får lavere pris. Ved å få lavere pris i de fysiske butikkene vil kundene kunne handle samme produkt til samme pris på nett og i butikk, men i butikk slippe å vente på frakt samt betale for porto.


% \subsection{Videre utvikling}

% - Hva mangler
% - Hva må gjøres
% - Hvilke andre funksjoner kan implementeres senere for å please kunder enda mer, men som vi ikke gjorde nå pga. spisset målgruppe?
\section{\textcolor[HTML]{D32F2F}{Oppsummering}}
Det viktigste vi lærte i prosjektet var hvor nyttig det var med brukertester.
brukere tenker annerledes enn de som lager
vite hvilke behov som finnes
tenke høyt under test
%Gruppen har samarbeidet godt gjennom prosessen, og

\subsection{Refleksjon om tverrfaglige grupper}
I gruppen som ble satt sammen var alle medlemmene informatikkstudenter. Gruppen ble dermed ikke like tverrfaglig som mange av de andre gruppene i emnet, men det fungerte stort sett greit. Det kunne selvfølgelig vært en fordel å for eksempel ha en student fra produktdesign når det skulle lages og printes poster, men gruppen løste oppgaven godt selv uten ressursene som mange andre grupper hadde.

\subsection{Retrospekt}
Dersom vi skulle gjort prosjektet en gang til er det noen ting vi ville gjort annerledes. For det første ville vi ha kjørt flere observasjoner, og vi burde ikke ha spisset inn så tidlig som vi gjorde.

\subsection{Videre arbeid med produkt}
Prototypen vi har kommet fram til skiller seg klart fra andre apper som finnes på markedet i dag, og vil gi brukerne en helt ny måte å handle på. Vi mener at dette absolutt kan være verdt å satse på. Det forutsetter at man får med seg handlesentrene på ideen, men vi tror at dette kan hjelpe mange kunder med å handle mer effektivt. På denne måten kan man sammenligne priser og produkter, og det blir også mulig for kunden å se antall på lager. Dette er noe mange brukere nevnte i spørreundersøkelsen vi gjorde, og det tyder på at dette er noe forbrukerne savner. Løsningen vi har laget vil også gjøre det raskt og lettvint for forbrukere på storhandel å handle. De trenger ikke bære på mange handleposer, eller gå rundt med stor handlevogn. Alle varene hentes til slutt, ferdig pakket, og kan bæres rett inn i bilen.
\\\\
Om butikkene vil se på dette som en god løsning er ikke sikkert, men vi tror at dersom man får kartlagt alle behov og finner en god løsning av alle aspekt vil dette kunne trekke mange kunder. Ved å ha en totalpris i bunn av handlelisten blir kunden observant på hva det koster, og det kan gjøre at kunden handler for mindre beløp. Det er det ikke sikkert butikkene liker. Det blir også langt vanskeligere med mersalg når brukeren ikke betaler i kassen i hver butikk, men kun til slutt for hele handleopplevelsen.
\\\\
Løsningen vi har kommet opp med krever en annen disponering av ressurser enn man ser i handlesentrene i dag. Med vår løsning vil man ikke trenge personell i kassen, men heller ute i butikken som kan hjelpe kundene. Det vil da være ved å finne riktig størrelse, besvare spørsmål om produktet og gi råd til hva man burde velge. 

